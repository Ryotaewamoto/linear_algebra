# 連立方程式の例

## 問題
果物の詰め合わせセットが2種類あります。一つ目の詰め合わせセットにはリンゴが2個とオレンジが8個入っていて560円です。もう一つの詰め合わせセットにはリンゴが5個とオレンジが3個入っていて550円です。リンゴとオレンジの単価はそれぞれいくらでしょうか。

### 変数と制約条件

変数の導入
- $x$ : リンゴの単価(円)
- $y$ : オレンジの単価(円)

### 制約条件
- $x \in \mathbb{Z} , x \geq 0$ $x$ は整数で負ではない
- $y \in \mathbb{Z} , y \geq 0$ $y$ は整数で負ではない

### 定式化

リンゴが2個とオレンジが8個入って560円

↓

$2x + 8y = 560$ 

リンゴが5個とオレンジが3個入って550円

↓

$5x + 3y = 550$ 

### 連立方程式

$$
\begin{cases} 2x + 8y &= 560 \\ 5x + 3y &= 550 \end{cases}
$$

ただし、$x,y \in \mathbb{Z}, x \geq 0, y \geq 0$ 

### 解答
- リンゴの単価ん: 80円
- オレンジの単価: 50円

### 連立方程式の解放
各方程式に含まれる変数のこすうを減らすように式変形を行います。式変形は必ず次に方法になります。

- 式全体に適切な定数をかける
- ある式に別の式の定数灰を足す、あるいは引く
- すでに求まっている変数を式の値に代入する

この方法を**掃き出し法**と言います。
これは古典的な連立方程式の解放です。

## オイラーの公式
オイラーの公式は以下のように与えられる。

$$ e^{i x} = \cos{x} + i \sin{x} $$

##  $\varepsilon - \delta$ 論法
任意の $\varepsilon > 0$ についてある $\delta > 0$ が存在して、任意の $x \in \mathbb{R}$ に対して $0 < |x - a| < \delta$ ならば $|f(x) - f(a)| < \varepsilon$ を満たすとき $f(x)$ は $a$ で連続であるという。
